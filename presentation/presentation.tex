\documentclass{beamer}

\usepackage{amsthm}
\usepackage{amsmath}
\usepackage{amssymb}
\usepackage{caption}
\usepackage{hyperref}
\usepackage{graphicx}
\usetheme{metropolis}

\title{Timetabling}
\author{Conor Newton}

\DeclareMathOperator{\obj}{obj}
\DeclareMathOperator{\students}{students}
\DeclareMathOperator{\staff}{staff}
\DeclareMathOperator{\ts}{timeslot}
\DeclareMathOperator{\room}{room}
\DeclareMathOperator{\prooms}{preferred\_rooms}
\DeclareMathOperator{\module}{module}
\DeclareMathOperator{\capacity}{capacity}
\DeclareMathOperator{\dept}{dept}
\DeclareMathOperator{\unv}{unavailable}
\DeclareMathOperator{\pathwayone}{pathway\_one}
\DeclareMathOperator{\act}{activities}
\DeclareMathOperator{\dayow}{day}
\DeclareMathOperator{\blame}{blame}


\usepackage{listings}

\lstnewenvironment{pseudo}[1][]{
	\minipage{\linewidth}
	\lstset{ %this is the stype
		keywordstyle=\color{blue}, %set keyword colour
		mathescape=true, %allow use of math mode within code
		keywords={FOR, IF, END, VAR, CONST, RETURN, DO, TRUE, FALSE, FUNCTION, AS, SUB, THEN, WHILE, NOT, ELSE}, %keywords for pseudo
		tabsize=4, %make whitespace readable
		basicstyle=\ttfamily\small, %use courier font
		title={#1}, %title takes first parameter
		frame=tB, %put it in a pretty box
		numbers=left, %force line numbers in left margin
		numberstyle=\ttfamily, %use courier font
	}
}
{\endminipage}


\begin{document}
\begin{frame}
	\titlepage
\end{frame}

\begin{frame}
	\frametitle{Introduction}
	
	What is a timetable?
	
	\begin{itemize}
		\item Given a set of activities, a set of rooms, and a range of times.
		\item A timetable is a choice of timeslot and room for each activity
	\end{itemize}
\end{frame}


\begin{frame}
	\frametitle{Introduction}
	In practice, we have many more requirements than this.
	
	Some other things to consider:
	
	\begin{itemize}
		\item Students
		\item Staff
		\item Room capacity
		\item Equipment Requirements
	\end{itemize}
\end{frame}

\begin{frame}
	\frametitle{Hard \& Soft Constraints}
	Hard-Constraints
	
	\begin{itemize}
		\item Our solution should have zero hard-constraint violations.
	\end{itemize}

	Soft-Constraints
	
	\begin{itemize}
		\item We aim to minimize the number of soft-constraint violations.
	\end{itemize}
	
\end{frame}

\begin{frame}
	\frametitle{Hard \& Soft Constraints Examples}
	Examples of Hard-Constraints
	
	\begin{itemize}
		\item Only one activity in a room at a time.
		
		\item The capacity of a chosen room must be greater than the number of students on an activity
		
		\item Equipment Requirements
	\end{itemize}
	
	Examples Soft-Constraints
	
	\begin{itemize}
		\item Student activity clashes
		\item Lunch Breaks
		\item Preferred rooms
	\end{itemize}
	
\end{frame}

\begin{frame}
	\frametitle{Our Approach}
	
	\begin{itemize}
		\item Stage 1 - Generate initial solution satisfying the hard constraints
		\item Stage 2 - Continuously make improvements to our existing solution
	\end{itemize}
\end{frame}

\begin{frame}
	\frametitle{Stage 1 - Backtracking}
	To generate our initial solution we make use of a backtracking algorithm.
	
	\begin{itemize}
		\item Incrementally builds a solution, by looping through the activities and assigning them timeslots \& rooms.
		\item When the solution cannot be extended any further it will "backtrack"
	\end{itemize}
	
\end{frame}

\begin{frame}
	\frametitle{Stage 2 - Optimization}
	What we need?
	
	\begin{itemize}
		\item An initial solution.
		\item Objective function.
		\item Neighbourhood operators.
		\item Optimization algorithm (Simulated Annealing/Tabu Search).
	\end{itemize}
	
\end{frame}


\begin{frame}
	\frametitle{Stage 2 - Optimization (Objective function)}
\begin{block}{}
	\begin{itemize}
		\item An objective function gives a score to a timetable (lower is better).
		
		\item It should consider soft-constraint violations
	\end{itemize}
\end{block}
	\begin{block}{}
	\begin{equation*}
	\obj(A) = w_1\cdot f_1(A) + w_2 \cdot f_2(A) + w_3 \cdot f_3(A)
	\end{equation*}
	\end{block}
\end{frame}

\begin{frame}
	\frametitle{Stage 2 - Optimization (Neighbourhood operators)}
	
	What are neighbourhood operators?
	
	\begin{itemize}
		\item Allows us to move to a different solution.
	\end{itemize}
	
	We make use of two neighbourhood operators
	
	\begin{itemize}
		\item Simple Swaps 
		\item Kempe Swaps
	\end{itemize}

	Both are commonly used together for timetabling problems.
\end{frame}

\begin{frame}
	\frametitle{Stage 2 - Optimization (Neighbourhood operators)}
	
	Two kinds of Simple Swaps
	
	\begin{itemize}
		\item Choose two random activities
		\item Swap their time and location
	\end{itemize}

	Alternatively,
	
	\begin{itemize}
		\item Choose a random activity and new free timeslot and room
		\item Move an activity to this room and timeslot
	\end{itemize}
\end{frame}

\begin{frame}
	\frametitle{Stage 2 - Optimization (Neighbourhood operators)}
	
	Kempe Swaps are a bit more involved...
	
		\begin{figure}[H]
		\centering
		\includegraphics[scale=0.75]{kempe_swap.png}
	\end{figure}

\end{frame}

\begin{frame}
	\frametitle{Stage 2 - Optimization (Solution Space)}
	
	\begin{figure}[H]
		\centering
		\includegraphics[scale=0.50]{solution_space.png}
	\end{figure}
	
\end{frame}

\begin{frame}
	\frametitle{Stage 2 - Optimization (Simulated Annealing)}
	
	What is simulated annealing?
	\begin{itemize}
		\item A metaheurstic for finding the global minimum of a function.
		\item Uses probability to avoid getting stuck in local minimums.
	\end{itemize}

	How it works:
	
	\begin{enumerate}
		
		\item We choose a random neighbouring timetable.
		\item If it's has a lower score, we set it as our current solution.
		\item Otherwise, we set it as our current solution with probability $\exp(\frac{current\_score - new\_score}{temp})$
		\item We set $temp = temp * cooling\_rate$
		\item We repeat these steps until $temp$ falls below a  value.
	\end{enumerate}
	
\end{frame}

\begin{frame}
	\frametitle{Results \& Implementation}
	
	\begin{itemize}
		\item Implemented in C++
		\item Produces a solution with an objective score of $< 1000$ in less than 10 minutes.
		\item Produces a solution with an objective score of $< 100$ after 35 minutes.
		\item The data set had about $18000$ students, $3000$ activities, $600$ rooms.
		\item Used a list of constraints that represented the university's requirements.
		
		\item There are still many more improvements for to be made!
	\end{itemize}
\end{frame}


\end{document}
	